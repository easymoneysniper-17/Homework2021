\documentclass{article}
\usepackage{mathrsfs}
\usepackage{bm}
\usepackage{amsmath}
\usepackage{amsthm}
\usepackage{amssymb}
\usepackage{graphicx}
\usepackage{color}
\usepackage{comment}
%\include{macros}
%\usepackage{floatflt}
%\usepackage{graphics}
%\usepackage{epsfig}


\theoremstyle{definition}
\newtheorem{theorem}{Theorem}[section]
\newtheorem{lemma}[theorem]{Lemma}
\newtheorem{proposition}[theorem]{Proposition}
\newtheorem{corollary}[theorem]{Corollary}

\theoremstyle{definition}
\newtheorem*{defition}{Definition}
\newtheorem*{example}{Example}

\theoremstyle{remark}
\newtheorem*{remark}{Remark}
\newtheorem*{note}{Note}
\newtheorem*{exercise}{Exercise}

\setlength{\oddsidemargin}{-0.25 in}
\setlength{\evensidemargin}{-0.25 in} \setlength{\topmargin}{-0.25
in} \setlength{\textwidth}{7 in} \setlength{\textheight}{8.5 in}
\setlength{\headsep}{0.25 in} \setlength{\parindent}{0 in}
\setlength{\parskip}{0.1 in}

\newcommand{\homework}[5]{
\pagestyle{myheadings} \thispagestyle{plain}
\newpage
\setcounter{page}{1} \setcounter{section}{#5} \noindent
\begin{center}
\framebox{ \vbox{\vspace{2mm} \hbox to 6.28in { {\bf
Part~II:~Applications~of~Network~Models~(Spring 2021) \hfill Homework: 04} }
\vspace{6mm} \hbox to 6.28in { {\Large \hfill #1 \hfill} }
\vspace{6mm} \hbox to 6.28in { {\it Lecturer: #2 \hfill} }
\vspace{2mm} \hbox to 6.28in { {\it \hspace{13mm} #3 \hfill} }
\vspace{2mm} \hbox to 6.28in { {\it Student: #4 \hfill} }
\vspace{2mm} } }
\end{center}
\markboth{#1}{#1} \vspace*{4mm} }


\begin{document}

\homework{Lecture 1: Shortest Path Problem}
{Junlong Zhang \hspace{5mm} {\tt zcs@mail.tsinghua.edu.cn}}
{}
{Zhenyu Jin \hspace{11mm} {\tt jzy20@mails.tsinghua.edu.cn}}{8}

\section*{P1.}
Give the conditions under which Dijkstra algorithm can work. Prove that under these conditions Dijkstra algorithm is correct, that is, we can indeed obtain the shortest paths by using Dijkstra algorithm.
\section*{Solution:}
\begin{itemize}
	\item The condition that Dijkstra algorithm can work is non negative weight edge exists in the network.
	\item Sufficient: \\
\quad \quad Given the condition above, then the distance of point to original must be increasing. According to the formulation given below, $d(i)=min{d(i),d(j)+l(i,j)}$, which point $j$ in the linking set of $i$. Suppose that we have already sort the points in the set, except origin point, the point are $n_0$, $n_1$,$\cdots$, $n_k$, i.e., $d(n_0)$ $\leq$ $d(n_1)$ $\leq$ $\cdots$ $\leq$ $d(n_m)$. Then we have that $d(n_i)=min{l(s,n_i),min_{j<i}{d(n_j)+l(n_i,n_j)}}$ .

\end{itemize}


\section*{P2.}
Give the conditions under which Floyd algorithm can work. Prove that under these conditions Floyd algorithm is correct, that is, we can indeed obtain the shortest paths by using Floyd algorithm.
\section*{Solution:}
\begin{itemize}
\item In this problem, we should restrain that $y,w\in Z$, then we have the model of a stochastic program with a mixed-integer second stage:
\begin{equation}
\begin{aligned}
min \quad &150x_1+230x_2+260x_3 \\
&-\frac{100}{3}(170w_{11}-238y_{11}+150w_{21}-210y_{21}+36w_{31}+10w_{41})\\
&-\frac{100}{3}(170w_{12}-238y_{12}+150w_{22}-210y_{22}+36w_{32}+10w_{42})\\
&-\frac{1}{3}(170w_{13}-238y_{13}+150w_{23}-210y_{23}+36w_{33}+10w_{43})\\
s.t. \quad &x_1+x_2+x_3\leq500, 3x_1+100y_{11}-100w_{11}\geq200 \\
&3.6x_2+100y_{21}-100w_{21}\geq 2.4, 100w_{31}+100w_{41}\leq24x_3,100w_{31}\leq6000 \\
&2.5x_1+100y_{12}-100w_{12}\geq200, 3x_2+100y_{22}-100w_{22}\geq240 \\               
&100w_{32}+100w_{42}\leq20x_3, 100w_{32}\leq6000,2x_1+y_{13}-w_{13}\geq200 \\   
&2.4x_2+y_{23}-w_{23}\geq240, w_{33}+w_{43}\leq16x_3,w_{33}\leq6000 \\
&x,y,w\geq0,y,w\in Z
\end{aligned}
\end{equation}
\end{itemize}


\section*{P3.}
Give a counterexample for which Floyd algorithm does not work.
\section*{Solution:}
\begin{itemize}
\item Denote $x_{ik}$: acres of land devoted to type $i$ during year $k$
\item $w_{ijk}$: tons of type $i$ sold during year $k$ in case $j$
\item $y_{ijk}$: tons of type $i$ purchased during year $k$ in case $j$
\item According to sugar beets cannot be planted two successive years on the same field, we have,
\begin{equation}
\begin{aligned}
min \quad &\sum_{k=1}^2 150x_{1k}+230x_{2k}+260x_{3k} \\                                                                      
& -\frac{1}{3}(170w_{111}-238y_{111}+150w_{211}-210y_{211}+36w_{311}+10w_{411} \\
& -\frac{1}{3}(170w_{121}-238y_{121}+150w_{221}-210y_{221}+36w_{321}+10w_{421} \\
& -\frac{1}{3}(170w_{131}-238y_{131}+150w_{231}-210y_{231}+36w_{331}+10w_{431} \\
& -\frac{1}{3}(170w_{112}-238y_{112}+150w_{212}-210y_{212}+36w_{312}+10w_{412} \\
& -\frac{1}{3}(170w_{122}-238y_{122}+150w_{222}-210y_{222}+36w_{322}+10w_{422} \\
& -\frac{1}{3}(170w_{132}-238y_{132}+150w_{232}-210y_{232}+36w_{332}+10w_{432} \\ 
s.t. \quad & x_{1k}+x_{2k}+x_{3k}\leq500,3x_{1k}+y_{11k}-w_{11k}\geq200 \\                                 
& 3.6x_{2k}+y_{21k}-w_{21k}\geq240,w_{31k}+w_{41k}\leq24x_{3k},w_{31k}\leq6000 \\
& 2.5x_{1k}+y_{12k}-w_{12k}\geq200,3x_{2k}+y_{22k}-w_{22k}\geq240 \\               
& w_{32k}+w_{42k}\leq20x_{3k},w_{32k}\leq6000,2x_{1k}+y_{13k}-w_{13k}\geq200 \\
& 2.4x_{2k}+y_{23k}-w_{23k}\geq240,w_{33k}+w_{43k}\leq16x_{3k},w_{33k}\leq6000 \\
& for k=1,2
& x_{32}\leq500-x_{31}
& x,y,w\geq0
\end{aligned}
\end{equation}
\end{itemize}


\section*{P4.}
Analyze the time complexity of Dijkstra algorithm and Bellman-Ford algorithm.
































\end{document}