\documentclass{article}
\usepackage{mathrsfs}
\usepackage{bm}
\usepackage{amsmath}
\usepackage{amsthm}
\usepackage{amssymb}
\usepackage{graphicx}
\usepackage{color}
\usepackage{comment}
%\include{macros}
%\usepackage{floatflt}
%\usepackage{graphics}
%\usepackage{epsfig}


\theoremstyle{definition}
\newtheorem{theorem}{Theorem}[section]
\newtheorem{lemma}[theorem]{Lemma}
\newtheorem{proposition}[theorem]{Proposition}
\newtheorem{corollary}[theorem]{Corollary}

\theoremstyle{definition}
\newtheorem*{defition}{Definition}
\newtheorem*{example}{Example}

\theoremstyle{remark}
\newtheorem*{remark}{Remark}
\newtheorem*{note}{Note}
\newtheorem*{exercise}{Exercise}

\setlength{\oddsidemargin}{-0.25 in}
\setlength{\evensidemargin}{-0.25 in} \setlength{\topmargin}{-0.25
in} \setlength{\textwidth}{7 in} \setlength{\textheight}{8.5 in}
\setlength{\headsep}{0.25 in} \setlength{\parindent}{0 in}
\setlength{\parskip}{0.1 in}

\newcommand{\homework}[5]{
\pagestyle{myheadings} \thispagestyle{plain}
\newpage
\setcounter{page}{1} \setcounter{section}{#5} \noindent
\begin{center}
\framebox{ \vbox{\vspace{2mm} \hbox to 6.28in { {\bf
Part~II:~Stochastic~Programming~(Spring 2021) \hfill Homework: 08} }
\vspace{6mm} \hbox to 6.28in { {\Large \hfill #1 \hfill} }
\vspace{6mm} \hbox to 6.28in { {\it Lecturer: #2 \hfill} }
\vspace{2mm} \hbox to 6.28in { {\it \hspace{13mm} #3 \hfill} }
\vspace{2mm} \hbox to 6.28in { {\it Student: #4 \hfill} }
\vspace{2mm} } }
\end{center}
\markboth{#1}{#1} \vspace*{4mm} }


\begin{document}

\homework{Lecture 2: Uncertainty and Modeling Issues}
{Junlong Zhang \hspace{5mm} {\tt zhangjunlong@tsinghua.edu.cn}}
{}
{Zhenyu Jin \hspace{11mm} {\tt jzy20@mails.tsinghua.edu.cn}}{8}


\section*{P1.}
Let $x$ represent the first-stage production of a given good. Let $\xi$ be the
demand for the same good. A typical second stage would consist of
selling as much as possible, namely, min($\xi$, $x$). Obtain a closed form
expression for the recourse function $E_{\xi}[min(\xi, x)]$ in the following
cases of $\xi$:
\begin{itemize}
    \item (a) Poisson distribution,
    \item (b) A normal distribution.
\end{itemize}
\subsection*{Solution:}
\begin{itemize}
    \item (a) Suppose that $\xi \sim P(\lambda)$
    \begin{equation*}
        E_{\xi}[min(\xi, x)] = \sum_{\xi=0}^{x} \xi e^{-\lambda}
         \frac{\lambda^{\xi}}{\xi!} + \sum_{\xi=x+1}^{\infty} x e^{-\lambda}
         \frac{\lambda^{\xi}}{\xi!} 
    \end{equation*}
    \item (b) Suppose that $\xi \sim N(\mu, \sigma^2)$
    \begin{equation*}
        \begin{aligned}
        E_{\xi}[min(\xi, x)] &= \frac{1}{\sqrt{2\pi}\sigma}(\int_{-\infty}^x 
                                \xi e^{-\frac{(\xi-\mu)^2}{2\sigma^2}} d\xi +
                                \int_x^{\infty} x e^{-\frac{(\xi-\mu)^2}{2\sigma^2}} d\xi)  \\
                             &= \frac{\sigma}{\sqrt{2\pi}}(1-e^{-\frac{(\frac{x-\mu}{\sigma})^2}{2}})
                                \frac{1}{\sqrt{2\pi}\sigma}[\mu \phi(x) + x(1-\phi(x))]
        \end{aligned}
    \end{equation*}
\end{itemize}


\section*{P2.}
Consider an airplane with $x$ seats. Assume passengers with reservations
show up with probability $0.90$, independently of each other.
\begin{itemize}
    \item (a) Let $x = 40$. If $42$ passengers receive a reservation, what is the
    probability that at least one is denied seat.
    \item (b) Let $x = 50$. How many reservations can be accepted under the
    constraint that the probability of seating all passengers who arrive
    for the flight is greater than 90$\%$ ?
\end{itemize} 

\subsection*{Solution:}
\begin{itemize}
	\item (a) Donate $Y$ as the number of customer shows up.
    \begin{equation*}
        \begin{aligned}
            P=P\{Y=41\}+P\{Y=42\}=42*0.9^{41} \times 0.1 +0.9^{42}
        \end{aligned}
    \end{equation*}
	\item (b)
    \begin{equation*}
        \begin{aligned}
            &P\{Y=y\} = C_{50}^y p^y(1-p)^{50-y} \\
            &E[Y] = \sum_0^{50} C_{50}^y yp^y(1-p)^{50-y}
        \end{aligned}
    \end{equation*}	
\end{itemize}


\section*{P3.}
Show that VaR is not a coherent risk measure.
\subsection*{Solution:}
\begin{itemize}
    \item The coherent risk measure has the following property 
according to the wikipedia: Sub-additivity.
    \item Sub-additivity: If $Z_1$, $Z_2$ $\in$ $\mathcal{L}$ ,
then $\varrho(Z_1 + Z_2)$ $\leq$ $\varrho(Z_1) + \varrho(Z_2)$, i.e., 
the risk of two portfolios together cannot get any worse than adding 
the two risks separately: this is the diversification principle.
    \item VaR obviously doesn't respect the sub-additivity property.
    \item According to the definition of the VaR: $VaR_{\alpha}(\xi)$=$min$\{$t$$|$$P(\xi \leq t)\geq \alpha$\}. 
    \item As a simple example to demonstrate the non-coherence of value-at-risk 
consider looking at the VaR of a portfolio at 95$\%$ 
confidence over the next year of two default-able zero coupon bonds that mature 
in 1 years time denominated in our numeraire currency. \\

    Assume the following:\\
    \begin{itemize}
        \item The current yield on the two bonds is 0$\%$
        \item The two bonds are from different issuers
        \item Each bond has a 4$\%$ probability of defaulting over the next year
        \item The event of default in either bond is independent of the other
        \item Upon default the bonds have a recovery rate of 30$\%$
    \end{itemize}
    Under these conditions the 95$\%$ VaR for holding either of the bonds 
        is 0 since the probability of default is less than 5$\%$. However if we held 
        a portfolio that consisted of 50$\%$ of each bond by value then the 95$\%$
        VaR is 35$\%$ (= 0.5*0.7 + 0.5*0) since the probability of at least one of 
        the bonds defaulting is 7.84$\%$ (= 1 - 0.96*0.96) which exceeds 5$\%$.
        This violates the sub-additivity property showing that VaR is not a coherent
        risk measure.
\end{itemize}


\section*{P4.}
Prove that CVaR is a coherent risk measure.
\subsection*{Solution:}
\begin{itemize}
    \item Let $F(x)$ be a probability distribution of a random variable $X$.
    \begin{equation*}
        \begin{aligned}
            F(x) = Prob{X \leq x} 
        \end{aligned}
    \end{equation*}
    and for some probability $q \in (0, 1)$ let’s define the q-quantile
    \begin{equation*}
        \begin{aligned}
            x_q = inf\{x|F(x) \geq q\}
        \end{aligned}
    \end{equation*}
    \item If $F(·)$ is continuous we have $F(x_q) = q$, while if $F(·)$ is discontinuous in $x_q$ and therefore $Prob{X = x_q} > 0$
    we may have $F(x_q)$ = $Prob{X \leq xq} > q$. Our definition of q-Tail Mean $x_q$ as the “expected value of the
    distribution in the q-quantile” must take this fact into account.
    \item Definition: For a random variable $X$ and for a specified level of probability $q$, let’s define the q-Tail Mean:
    \begin{equation*}
        \begin{aligned}
            \overline{x_q} & \equiv \frac{1}{q} E\{X 1_{X\leq x_q} \}	
            + (1 -\frac{F(x_q)}{q})x_q \\
            & =\frac{1}{q} E\{X 1_{X\leq x_q}^{q} \}
        \end{aligned}
    \end{equation*}  
    where in the last expression we introduced
    \begin{equation*}
        \begin{aligned}
            1_{X\leq x_q}^{q} = 1_{X\leq x_q} + \frac{q-F(x_q)}{Prob{X = x_q}}1_{X = x_q}^{q}
        \end{aligned}
    \end{equation*}  
    \item The second term in the sum is zero if $Prob{X = x_q} = 0$. In what follows we will make use of the following
    properties:
    \begin{equation*}
        \begin{aligned}
            & E\{X 1_{X\leq x_q}^{q} \} = q \\
            & 0 \leq 1_{X\leq x_q}^{q} \leq 1
        \end{aligned}
    \end{equation*}  
    \item The only thing to show is in the case $X = x_q$:
    \begin{equation*}
        \begin{aligned}
            1_{X\leq x_q}^{q} |_{X = x_q}= 1 + \frac{q-F(x_q)}{Prob{X = x_q}} = \frac{q-F(x_q)}{Prob{X = x_q^-}} \in [0,1]
        \end{aligned}
    \end{equation*} 
\end{itemize}



\section*{P5.}
Prove that CVaR is an upper bound of VaR.
\subsection*{Solution:}
\begin{equation*}
    \begin{aligned}
        CVaR_{\alpha} &= E(X|X\geq q_{\alpha}) = \frac{E(X1_{X \geq x_q})}
                        {Prob\{X \geq x_q\}}  \\ 
                      &= \frac{E(X-q_{\alpha}+q_{\alpha}1_{X\geq q_{\alpha}})}{Prob\{X \geq x_q\}} \\
                      &= \frac{E((X- q_{\alpha})1_{X \geq x_q})}
                      {Prob{X \geq x_q}}  + q_{\alpha} \frac{E(1_{X \geq x_q})}{Prob\{X \geq x_q\}}\\ 
                      &= \frac{E((X-q_{\alpha})^+)}{\alpha} + q_{\alpha} \\
                      &\geq q_{\alpha} \\
                      &= VaR_{\alpha}
    \end{aligned}
\end{equation*}  





























\end{document}